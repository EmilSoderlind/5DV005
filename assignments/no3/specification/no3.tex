\documentclass[a4paper,12pt]{article}
\usepackage[left=2.5cm,right=2.5cm,top=2.5cm,bottom=2.5cm]{geometry} 
\usepackage{amsmath,amssymb,amsthm,algorithm,algorithmic,graphicx,yhmath,url,enumitem,lscape,hyperref,enumitem}
\usepackage{wrapfig,subfigure}
\newcounter{problem}
\newcounter{remark}
\newcounter{hint}
\newenvironment{remark}{\refstepcounter{remark} \vspace{0.1cm} \par \noindent {\bf Remark \arabic{remark}}}{\vspace{0.3cm}}
\newenvironment{hint}{\refstepcounter{hint} \vspace{0.1cm} \par \noindent {\bf Hint \arabic{hint}}}{\vspace{0.3cm}}
\newcommand{\R}{\mathbb{R}}
\newcommand{\N}{\mathbb{N}}
\newcommand{\Rn}{\mathbb{R}^n}
\newcommand{\Rnn}{\mathbb{R}^{n \times n}}
\newcommand{\bes}{\begin{equation*}}
  \newcommand{\ees}{\end{equation*}}
\newcommand{\be}{\begin{equation}}
  \newcommand{\ee}{\end{equation}}
\newcommand{\bms}{\begin{multline*}}
  \newcommand{\emults}{\end{multline*}}

% Matrices
\newcommand{\bbm}{\begin{bmatrix}}
  \newcommand{\ebm}{\end{bmatrix}}
\newcommand{\bpm}{\begin{pmatrix}}
  \newcommand{\epm}{\end{pmatrix}}

% Strecthing
\renewcommand{\arraystretch}{1.3}

\newcommand{\eps}{\epsilon}
\newcommand{\fl}{\text{fl}}
\newcommand{\Lp}{{L^p}}
\newcommand{\Ker}{\text{Ker}\,}
\newcommand{\loc}{{\text{loc}}}
\newcommand{\ccinf}{C_c^\infty}
\newcommand{\supp}{\text{supp}}
\newcommand{\dist}{\text{dist}}

% Gravitational acceleration
\newcommand{\bg}{\mathbf{g}}
% Friction force
\newcommand{\bFf}{\mathbf{F_f}}
% Gravitational force
\newcommand{\bFg}{\mathbf{F_g}}
% Force
\newcommand{\bF}{\mathbf{F}}
% Position, velocity, acceleration
\newcommand{\br}{\mathbf{r}}
\newcommand{\bv}{\mathbf{v}}
\newcommand{\ba}{\mathbf{a}}

% Jacobian
\newcommand{\bDF}{\mathbf{DF}}

\newtheorem{theorem}{Theorem}[section]
\newtheorem{proposition}[theorem]{Proposition}
\newtheorem{lemma}[theorem]{Lemma}
\newtheorem{definition}[theorem]{Definition}
\title{Accuracy in artillery computations}
\author{Carl Christian Kjelgaard Mikkelsen}

\begin{document} 

\pagenumbering{arabic}

\thispagestyle{empty}

\noindent
Ume\aa{} University \hfill Fall 2018 \\
Department of Computing Science\\

\vskip 2.5cm

\begin{center} {\Huge {\bf Project 3}}\\{\LARGE Error estimation for artillery computations}\end{center} \vskip 0.3cm
\begin{center}
  {\huge Scientific Computing}
  \vfill
  {\Large The deadline for this project can be found at: \href{http://www8.cs.umu.se/kurser/5DV005/HT18/planering.html}{http://www8.cs.umu.se/kurser/5DV005/HT18/planering.html}\\
    (Link \emph{Overview} on the course homepage.)}
\end{center}
\vfill
{% \large
  \begin{itemize}
  \item The submission should consist of:
    \begin{itemize}
    \item The complete report, including
      \begin{itemize}
      \item A front page with the following information:
        \begin{enumerate}
        \item Your {\bf name}.
        \item The {\bf course name}.
        \item Your {\bf username} at the Department of Computing
          Science.
        \item The {\bf project number}.
        \item The {\bf version} of the submission (in case of re-submissions).
        \end{enumerate}
      \end{itemize}
    \item An appendix with the source code.
    \item To simplify feedback, the main report (optionally excluding
      the appendix) must have {\bf numbered sections} and {\bf page
        numbers}.
    \end{itemize}
  \item The submitted code must be {\tt MATLAB}-compatible. If you choose to
    work in Octave, verify that your code is {\tt MATLAB}-compatible before
    you submit your project.
  \item If you write your report using \LaTeX, double-check that your
    references have been resolved correctly before you submit. ``Figure ??'' is useless to any reader.
  \item Your report should be submitted as a pdf file uploaded via
    the\linebreak
    \href{https://www8.cs.umu.se/~labres/py/handin.cgi}{https://www8.cs.umu.se/\textasciitilde{}labres/py/handin.cgi}
    page, also available as the
    \begin{center}
      \href{https://www8.cs.umu.se/~labres/py/handin.cgi}{Submit/Check results}
    \end{center} link at the bottom left of the course home page.
  \item Furthermore, the source code should be available in a folder
    called {\tt edu/5dv005/assN} in your home folder, where N is the
    project number. You will probably have to create the folder
    yourself.
  \end{itemize}
}
\vfill

\newpage

\maketitle
\tableofcontents


\section{Primary purpose}

The primary purpose of this assignment to develop the ability to compute error estimates which are reliable and accurate. 

\section{Asymptotic error expansions}

We have considered the problem of approximating the range of a gun, the flight time of a shell or the elevation necessary to hit a particular target. It remains to compute reliable error estimates for such approximations.

In this project, we view each approximation $A$ as a function of the size of the time step $h$ used when computing the trajectories, i.e., $A=A_h$. We will investigate, if there exists asymptotic error expansions of the form
\be \label{equ:expansion}
T - A_h = \alpha h^p + \beta h^q + O(h^r), \quad 0 < p < q < r
\ee
The term $\alpha h^p$ is called the primary error term, while $\beta h^q$ is the secondary order term. We will obtain reliable error estimates.

The asymptotic error expansion describes the difference between the target value $T$ and the approximation $A_h$. We can not hope to obtain the exact value of $A_h$. The very best we can hope for is the floating point representation of $A_h$, i.e. $\hat{A}_h = \text{fl}(A_h)$, but in general this is not a realistic goal. The difference between $A_h$ and the computed value of $\hat{A}_h$ is the result of many rounding errors. By monitoring the \emph{computed} value of Richardson's fraction we can determine when the rounding error $A_h - \hat{A}_h$ is irrelevant compared with the error $T-A_h$.

\section{Software}

The function {\tt range\_rkx} moves the shell from point to point using a time step which is fixed except for the very last time step. Here a non-linear solver is used to compute the time step which will place the shell on the ground. This equation is solved to the limit of machine precision, specifically the tolerance passed to the underlying bisection routine is  $\text{tol} = 2^{-53}$. The function {\tt range\_rkx\_sabotage} is identical to {\tt range\_rkx} except that the tolerance is much larger, specifically, $\text{tol} = 2^{-3}$. The function {\tt a3int} computes the integral of a given function along a trajectory computed by {\tt range\_rkx}.

\section{Questions}

\begin{figure}
\begin{verbatim}
     k |    Approximation A_h | Fraction F_h |   Error estimate E_h
     1 |   2.895480163672e+00 |   0.00000000 |   0.000000000000e+00
     2 |   2.805025851403e+00 |   0.00000000 |  -9.045431226844e-02
     3 |   2.761200888902e+00 |   2.06399064 |  -4.382496250165e-02
     4 |   2.739629445828e+00 |   2.03161941 |  -2.157144307422e-02
     5 |   2.728927822736e+00 |   2.01571695 |  -1.070162309156e-02
     6 |   2.723597892360e+00 |   2.00783544 |  -5.329930376490e-03
     7 |   2.720938129638e+00 |   2.00391198 |  -2.659762721350e-03
     8 |   2.719609546672e+00 |   2.00195456 |  -1.328582965698e-03
     9 |   2.718945579511e+00 |   2.00097692 |  -6.639671619268e-04
    10 |   2.718613676976e+00 |   2.00048837 |  -3.319025345263e-04
    11 |   2.718447745963e+00 |   2.00024413 |  -1.659310128161e-04
    12 |   2.718364785520e+00 |   2.00012206 |  -8.296044325107e-05
    13 |   2.718323306559e+00 |   2.00006078 |  -4.147896106588e-05
    14 |   2.718302567373e+00 |   2.00002842 |  -2.073918585666e-05
    15 |   2.718292197853e+00 |   2.00001403 |  -1.036952016875e-05
    16 |   2.718287013122e+00 |   2.00001123 |  -5.184730980545e-06
    17 |   2.718284420669e+00 |   1.99993264 |  -2.592452801764e-06
    18 |   2.718283124268e+00 |   1.99973060 |  -1.296401023865e-06
    19 |   2.718282476068e+00 |   2.00000000 |  -6.482005119324e-07
    20 |   2.718282151967e+00 |   2.00000000 |  -3.241002559662e-07
\end{verbatim} \caption{The output of {\tt a3f2.m} after the completion of {\tt MyRichardson.m}} \label{fig:a3f2}
\end{figure} 

\begin{enumerate}
\item Copy {\tt scripts/a3f1.m} in {\tt work/MyRichardson.m} and complete the function according to the the specification. It is likely, that the function is working correctly when the corresponding minimal working example {\tt scripts/a3f2.m} returns the output given by Figure \ref{fig:a3f2}.


\item \label{q:range} Develop a script {\tt a3range.m} which apply Richardson's techniques to compute the range of the shot whose physical parameters are given by {\tt a3f3.m}

  \begin{itemize}
  \item Use methods {\tt 'rk1', 'rk2', 'rk3', 'rk4'} to compute the trajectories
  \item Use time steps $h_k = 2^{-k}$ seconds for $k=0,1,2,\dotsc$.
  \end{itemize}
  For each of the four methods:
  \begin{enumerate}
  \item Determine the power of the primary error term.
  \item Determine the power of the secondary error term.
  \item Determine when the computed value of Richardson's fractions behaving in a manner consistent with an asymptotic error expansion of the type given by equation \eqref{equ:expansion}.
  \item Identify the best approximation of the range and explain why the error estimate is reliable.
  \end{enumerate}

\item \label{q:time} Develop a script {\tt a3time} which compute the flight time of the shot given by {\tt a3f3} using your method of choice.
  \begin{enumerate}
  \item You must include an error estimate.
  \item You must explain why your error estimate can be trusted.
  \item You must discuss the behavior of Richardson's fractions.
  \end{enumerate}

\item \label{q:elevation} Develop a script {\tt a3low} which computes the low firing solution for a target located at $15000$ meters to the right of the gun given by {\tt a3f3.m}.
  \begin{itemize}
  \item You must include an error estimate.
  \item You must explain why your error estimate can be trusted.
  \item You must compute Richardson's fractions and discuss their behavior.
  \end{itemize}
  {\bf Warning:} Richardson's fraction will not behave correctly unless the elevations are computed with what would appear to be excessive accuracy! Expect to use residuals which are as small as $10^{-10}$ meters.
  {\bf Warning:} This is not a fast calculation, so begin with a small number of rows, say, 6 rows, in Richardson's table and see if this is enough.

\item \label{q:smooth} Develop a script {\tt a3range\_g7} which computes the range of the shot defined by {\tt a3f4}. 
\begin{enumerate}
\item For which methods do you retain the ability to estimate the range accurately?
\end{enumerate}

\item \label{q:sabotage} Develop a script {\tt a3range\_sabotage} which uses {\tt range\_rkx\_sabotage} to compute the range of the shot given by {\tt a3f3}. 
\begin{enumerate}
\item For which methods do you retain the ability to estimate the range accurately?
\end{enumerate}

\item \label{q:length} The script {\tt a3length} computes the length of the trajectory of the shot given by {\tt a3f3.m}.
  \begin{itemize}
  \item It uses methods {\tt 'rk1', 'rk2', 'rk3', 'rk4'} to compute the trajectories
  \item It uses time steps $h_k = 2^{-k}$ seconds for $k=0,1,2,\dotsc$.
  \item It uses the trapezoidal rule to compute the length of each trajectory.
  \end{itemize}
  \begin{enumerate}
  \item For each method what is the order of the primary error term?
  \end{enumrate}
\end{enumerate}


\section{Concluding remarks}

Our entire analysis hinges on the existence of an asymptotic error expansion (AEE), see equation \eqref{equ:expansion}. It can be difficult to prove the existence of an AEE, but close observation of the Richardson's fraction will allow you to determine when Richardson's error estimate is reliable, i.e., Questions \ref{q:range}, \ref{q:time}, \ref{q:elevation}.

Mathematically, the existence of an AEE requires a certain degree of differentiability to sustain the necessary Taylor expansions. In particular, higher order methods such as 'rk3' and 'rk4' require more differentiability than lower order methods such as 'rk1' and 'rk2'. The drag function used by {\tt a3f4} is a piecewise cubic polynomial which is two, but not three times differentiable. This significantly reduces the amount of differentiability available and causes the problems which you detected in Question \ref{q:smooth}.

Moreover, all central equations must solved accurately or you loose the ability to estimate the error accurately. The error made by {\tt range\_rkx\_sabotage} when computing the final time-step destroyed you ability to estimate the difference between the true range $r$ and the approximation $r_h$, see Question \ref{q:sabotage}. Similarly, it was necessary to use a very small residual when computing the elevation $\theta_h$, see Question \ref{q:elevation}.

The calculation of the length of the trajectory of the shell in Question \ref{q:length} illustrates a fundamental principle of scientific computing. In general, a calculation is only as a accurate as its least accurate component. If we use a first order method to compute the trajectory, then we gain nothing from using a second order method to compute the arc length. Similarly, if we use a third or fourth order method to compute the trajectory, then a second order accurate calculation of the arc length will reduce the overall accuracy to second order.

\end{document}