\documentclass[a4paper,12pt]{article}
\usepackage[left=2.5cm,right=2.5cm,top=2.5cm,bottom=2.5cm]{geometry} 
\usepackage{color}
\usepackage[usenames,dvipsnames]{xcolor}
\usepackage{amsmath,amssymb,amsthm,algorithm,algorithmic,graphicx,yhmath,url,enumitem,lscape,mathtools}
\usepackage{wrapfig,subfigure}

\newcounter{problem}
\newenvironment{problem}{\refstepcounter{problem} \noindent {\bf Problem \arabic{problem}}}{\newpage}
\newenvironment{solution}{\vspace{0.3cm} \par \noindent {\bf Solution}}{}
\newenvironment{verification}{\vspace{0.3cm} \par \noindent {\bf Verification}}{}
\newenvironment{hint}{\vspace{0.3cm} \par {\bf Hint:}}{}

\newcounter{remark}
\newenvironment{remark}{\refstepcounter{remark} \vspace{0.3cm} \par \noindent {\bf Remark \arabic{remark}}}{\vspace{0.3cm}}
\newcounter{lesson}
\newenvironment{lesson}{\refstepcounter{lesson} \vspace{0.3cm} \par \noindent {\bf Lesson \arabic{lesson}}}{\vspace{0.3cm}}
\newcommand{\R}{\mathbb{R}}
\newcommand{\N}{\mathbb{N}}
\newcommand{\Rn}{\mathbb{R}^n}
\newcommand{\Rnn}{\mathbb{R}^{n \times n}}
\newcommand{\bes}{\begin{equation*}}
\newcommand{\ees}{\end{equation*}}
\newcommand{\be}{\begin{equation}}
\newcommand{\ee}{\end{equation}}
\newcommand{\eps}{\epsilon}
\newcommand{\fl}{\text{fl}}

\begin{document}

\title{5DV005, Fall 2018, Lab session 4}
\author{Carl Christian Kjelgaard Mikkelsen}

\maketitle
\tableofcontents

\section{The time and the place}
The lab session will take place on
\begin{center}
Wednesday, November 28th, 2018, (kl. 13.00-16.00), Room MA416-426.
\end{center}

\section{The problems}

\begin{center}
  {\bf It is a smashing idea to check the folder {\tt lab4/scripts} carefully!}
\end{center}

\begin{problem}
  \begin{enumerate}
  \item Develop a function {\tt MySinh} which computes the function
    \bes
    f(x) = \frac{e^x - e^{-x}}{2}.
    \ees
    You may use the built-in function {\tt exp} to evaluate $e^x$ when $|x|$ is sufficiently large, but you must rewrite $f$ to avoid the subtractive cancellation at when $x \approx 0$.
  \item Develop a minimal working example {\tt MySinhMWE} which compares your implementation to the built-in function {\tt sinh}. It is possible to reduce the relative error below $10^{-15}$ on the interval $[-3,3]$.
  \end{enumerate}
\end{problem}

\begin{problem}
% Newton's method for square root with good initial guess
% MyNSqrt
  \begin{enumerate}
    \item Develop a function {\tt MyNewtonSqrt} which uses Newton's method for computing square roots subject. Your function must use the initial guess
    \bes
    x_0(s) = \frac{1}{3} s  + \frac{17}{24}
    \ees
    for $\sqrt{s}$ when $s \in [1,4]$.
  \item Develop a minimal working example {\tt MyNewtonSqrtMWE1} which compares your implementation to the built in function {\tt sqrt}. It is possible to reduce the relative error to $2u$ on the interval $[10^{-3}, 10^3]$.
  \end{enumerate}
\end{problem}

\begin{problem}
  \begin{enumerate}
  \item  Develop a function {\tt MyLog} which uses Newton's method to solve the non-linear equation $f(x) = 0$ where $f(x) = \exp(x) - \alpha$ and $\alpha > 0$. Your function must exploit the fact that if $\alpha = f \cdot 2^e$, then
  \bes
  \log(\alpha) = \log(f) + e \log(2)
  \ees 
 Your are free to use the special function {\tt log2} to determine $f$ and $e$. You are free to use the built-in value of $\log(2)$. Your function must use the initial guess
  \bes
  x_0(s) = a s + b, \qquad a = \log(2), \qquad b = - \frac{a + \log(a) + 1}{2}
  \ees
  for $\log(s)$ when $s \in [1,2]$.
\item Develop a minimal working example {\tt MyLogMWE} which compares your implementation to the built-in function {\tt log}. It is possible to reduce the relative error below $2u$ on the interval $[2,10]$.
\end{enumerate}
\end{problem}

\begin{problem}
% Develop a robust version of the secant method and the bisection method
% MyRobustSecant
  \begin{enumerate}
  \item Copy the script {\tt scripts/l4p4.m} into the function {\tt work/MyRobustSecant.m} and complete the function according to the specification.
  \item Develop a minimal working example {\tt MyRobustSecantMWE} which solves the your favorite non-linear equation. I recommend computing a firing solution.
    \end{enumerate}
\end{problem}

\bibliographystyle{plain}
  \bibliography{../../../lecture-notes/refs}
  
\end{document}






